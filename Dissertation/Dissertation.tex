\documentclass[a4paper,12pt]{article}

\relax
% Encoding and language support 
\usepackage[utf8]{inputenc} % Input encoding
\usepackage[T2A]{fontenc} % Font encoding for Cyrillic (Bulgarian)
\usepackage[bulgarian, english]{babel} % Language support

\usepackage{amsmath, amssymb}
\usepackage{listings} 
\usepackage{xcolor}
\usepackage{url}

%Comments
\usepackage{pdfcomment}

%Coding
\usepackage{listings}
\usepackage{xcolor} % Required for colors

% Define colors similar to Pygments/minted
\definecolor{codegreen}{rgb}{0,0.6,0}
\definecolor{codegray}{rgb}{0.5,0.5,0.5}
\definecolor{codepurple}{rgb}{0.58,0,0.82}

\definecolor{backcolour}{rgb}{0.95,0.95,0.92}

% Listing style
\lstset{
    commentstyle=\color{codegreen},
    keywordstyle=\color{magenta},
    numberstyle=\tiny\color{codegray},
    stringstyle=\color{codepurple},
    basicstyle=\ttfamily\footnotesize,
    breakatwhitespace=false,         
    breaklines=true,                 
    captionpos=b,                    
    keepspaces=true,                 
    numbers=left,                    
    numbersep=5pt,                  
    showspaces=false,                
    showstringspaces=false,
    showtabs=false,                  
    tabsize=2,
    frame=none,
}




\title{\textbf{Дисертация на тема
        \\[0.5em] \large{Разработване на виртуална среда за пенсионни актюерски изчисления в Python}}}
\author{Георги Веселинов Бурнаски
        \\ Катедра Приложна Математика
        \\ Факултет по Математика и Информатика
        \\ Софийски Университет "Св. Климент Охридски"}
\date{\today}

\begin{document}
\maketitle
\newpage
\section{Увод}
Актюерската математика заема все по-значимо място в съвременното общество, където управлението на финансовите рискове и дългосрочната устойчивост на социалните системи са от първостепенно значение.
\newline
\newline
В условията на застаряващо население, динамични финансови пазари и нарастващи изисквания към пенсионните системи, надеждните и прецизни актюерски изчисления се превръщат в ключов инструмент за вземане на информирани решения. Те подпомагат както държавните институции при моделиране и реформиране на пенсионните схеми, така и частните компании при разработване на застрахователни и инвестиционни продукти \cite{dalriada2023}.
\newline
\newline
Актюерските методи намират широко приложение в различни сфери: пенсионното осигуряване, животозастраховането здравното застраховане, управлението на фондове и финансовото планиране. Общото между всички тези области е необходимостта от точни прогнози, основани на статистически модели и вероятностни методи, които да оценяват бъдещи парични потоци, продължителност на живота и свързаните с тях рискове \cite{milliman2022}.
\newline
\newline
В наши дни пенсионните системи се изправят пред редица сериозни предизвикателства. Демографските промени, и по-специално увеличаването на средната продължителност на живота и намаляването на раждаемостта, водят до нарастващо съотношение между пенсионери и
активно работещо население \cite{bostonfed1999}. Това поставя под натиск публичните пенсионни фондове и създава необходимост от по-прецизни модели за оценка на бъдещите задължения. Паралелно с това колебанията на финансовите пазари и инфлационните процеси влияят върху доходността на пенсионните активи и изискват по-гъвкави подходи при управлението на риска \cite{ncpers2023}.
\newline
\newline
Съвременните технологии и програмни езици като Python предлагат нови възможности за прилагане на актюерската математика в практиката. Разработването на специализирана библиотека за пенсионни актюерски изчисления има двойна значимост: от една страна, улеснява изследователите и практиците при прилагането на сложни модели, а от друга – допринася за повишаване на прозрачността, възпроизводимостта и достъпността на тези изчисления \cite{lifelib2024, hyperexponential2023}.
\newline
\newline
Въпреки наличието на различни софтуерни решения за актюерски изчисления, много от тях са или твърде специализирани, или не предлагат необходимата гъвкавост и интеграция с други аналитични инструменти. Python, с богатия си екосистем от библиотеки за научни изчисления и машинно обучение, се явява като идеална платформа за разработване на такава библиотека. Тя би могла да обедини теоретичните основи на актюерската математика с практическите нужди на потребителите, предоставяйки лесен за използване и разширяем инструмент \cite{numpy2024, pandas2024}.
\newline
\newline
Разработването на библиотека за пенсионни актюерски изчисления в Python би могло да включва функции за изчисляване на настояща стойност на бъдещи парични потоци, моделиране на демографски и финансови рискове, симулации на сценарии и оптимизация на пенсионни стратегии. Освен това, интеграцията с други библиотеки за визуализация и анализ би позволила по-добро представяне и интерпретация на резултатите \cite{matplotlib2024, seaborn2024}.
\newline
\newline
В заключение, създаването на специализирана библиотека за пенсионни актюерски изчисления в Python представлява важна стъпка към модернизацията и усъвършенстването на актюерската практика. Тя би могла да подпомогне както академичните изследвания, така и практическите приложения, като предостави мощен и достъпен инструмент за анализ и управление на пенсионните системи в съвременния свят.
\newline
\newline
Пенсионната система в България следва модела на много развити страни и се състои
от три компонента, известни като "трите стълба":
\begin{itemize}
        \item Първи стълб: Държавно задължително пенсионно осигуряване (солидарност между поколенията). Това е държавната пенсия, която се финансира от текущите осигурителни вноски на работещите. Тя е задължителна за всички работещи.
        \item Втори стълб: Допълнително задължително пенсионно осигуряване (ДЗПО). Това е индивидуално натрупване на средства в избрани от самия осигурено лице пенсионни фондове. Той също е задължителен за хората, родени след 31.12.1959 г., които не са избрали да останат само в Първи стълб.
        \item Трети стълб: Допълнително доброволно пенсионно осигуряване (ДДПО). Това е напълно доброволно допълнително пенсионно осигуряване, при което хората сами решават да спестяват допълнително за своята пенсия.
\end{itemize}
В този труд ще се фокусираме върху изчисленията, свързани със втори и трети стълб от пенсионносигурителната система, тъй като те са поети от частните пенсионноосигурителни дружества. В България 20\% от доходите на работещите се отделят за пенсионно осигуряване, като 12.8\% отиват за първи стълб, а 5\% - за втори стълб. Важно е да се отбележи, че тези проценти могат да варират в зависимост от законодателството и икономическите условия. Останалите 2.2\% са за здравно осигуряване и трудови злополуки.\cite{ZDZPO_2004, ZDOO_2000}
\newline
\newline
Условията за пенсионните изчисления са регламентирани от различни нормативни актове, като основните от тях са:
\begin{itemize}
        \item Кодекс за социално осигуряване (КСО)
        \item Закон за допълнително задължително пенсионно осигуряване (ЗДЗПО)
        \item Закон за допълнително доброволно пенсионно осигуряване (ЗДОО)
        \item Наредба № Н-8 от 2004 г. за определяне на методиката за изчисляване на пенсиите
        \item Наредба № 3 от 2003 г. за условията и реда за изплащане на пенсиите от частните пенсионни фондове
\end{itemize}
Като изчисленията трябва да отговарят на изискванията, заложени наредба 69 от 2003 г. за определяне на методиката за изчисляване на пенсиите от частните пенсионни фондове. \cite{ZDZPO_2004, ZDOO_2000, DKFN_Pensions, NOI_Official}
\newline
\newline
Целта на настоящата дисертация е именно изгражданетона такава библиотека, която да обедини теоретичните основи на пенсионната математика с предимствата на съвременните програмни среди. Чрез нея се цели създаването на гъвкав инструмент, който да подкрепя както научната работа, така и практическите решения в областта на пенсионното осигуряване.
\newpage
\section{Математика}
\subsection{Вероятност за смърт}
Вероятността за смърт $q_x$ е вероятността човек на възраст $x$ да умре преди да достигне възраст $x+1$. Тази вероятност се взима от статистическата статистика на държавната осигурителна институция  в случая Националният осигурителен институт(НОИ)
\subsection{Вероятност за преживяване}
Вероятността за преживяване $p_x= 1-q_x$ е вероятността човек на възраст $x$ да оцелее до следващата година, т.е. до възраст $x+1$.
\subsection{Среден брой живи хора}
Средният брой живи хора на възраст $x$ се обозначава с $l_x$ и представлява броя на хората, които са живи на тази възраст в дадена популация. Тази стойност се използва за изчисляване на вероятността за смърт и други актюерски изчисления. Изчислява се по формулата:
\[l_{x} = l_0 \cdot \prod_{i=0}^{x-1} p_i\]
където $l_0$ е началният брой живи хора (на възраст 0), а $p_i$ е вероятността за преживяване на възраст $i$ и $l_x$ е броят на живите хора на възраст $x$.
\subsection{Дисконтиращ фактор}
Дисконтиращият фактор $v$ се използва за изчисляване на настоящата стойност на бъдещи парични потоци. Той се изчислява по формулата:
\[v = \frac{1}{1+i}\]
където $i$ е годишният лихвен процент зададен от пенсионноосигурителната компания. В нашият случай ще използваме техническия лихвен процент на пенсионна компания "Съгласие" който е 0.15\% \cite{DKFN_Pensions}
\subsection{Настояща стойност}
Настоящата стойност ($D_x$) на бъдещ паричен поток се изчислява чрез дисконтиране на този поток към настоящия момент. Формулата за изчисляване на настоящата стойност е:
\[D_x = l_x\cdot v^x\]
\subsection{Кумолативна настояща стойност}
Кумулативната настояща стойност ($N_x$) представлява сумата от настоящите стойности на всички бъдещи плащания за период от $n$ години. Тя се изчислява по формулата:
\[N_x = \sum_{i=x}^{max}D_x\]
\subsection{Актюерски фактори и изчисления на пенсии}
Актюерските фактори се използват за изчисляване на различни видове пенсии и други финансови продукти. Те включват:
\begin{itemize}
        \item \textbf{Проста пожизнена пенсия (фактор $k_1$)} - изчислява се като:
              \[k_1 = 12\cdot\left(\frac{N_{x}}{D_{x}-\frac{11}{24}}\right)\]
              където $N_x$ е кумулативната настояща стойност на бъдещите плащания, а $D_x$ е настоящата стойност на бъдещия паричен поток. Това представлява фактор, който се използва за изчисляване на месечната сума на Пожизнена пенсия. Която се изчислява като:
              \[P = \frac{S}{k_1}\]
              където $P$ е месечната сума на пенсията, а $S$ е натрупаната сума в пенсионния фонд към момента на пенсиониране.
        \item \textbf{Пожизнена пенсия с  период на гарантирано изплащане (фактор $к_2$)} - изчислява се като:
              \[k_2 = 12\cdot\left(\frac{N_{x+d}}{D_x} - \frac{11}{24}\cdot D_{x+d}{D_{x}}\right)+\frac{1-v^n}{1-\sqrt[12]{v}}\]
              където $N_x$ е кумулативната настояща стойност на бъдещите плащания, $N_{x+n-1}$ е кумулативната настояща стойност на бъдещите плащания след изтичане на гарантирания период от $n$ години, а $D_x$ е настоящата стойност на бъдещия паричен поток. Това представлява фактор, който се използва за изчисляване на месечната сума на Пожизнена пенсия с гарантиран период. Която се изчислява като:
              \[P = \frac{S}{k_2}\]
        \item \textbf{Допълнителна пожизнена пенсия полагаща се след период на разсрочено изплащане (фактор $k_3$)} - изчислява се като:
              \[k_3 = 12\cdot\left(\frac{N_{x+d}}{D_{x}} - \frac{11}{24}\cdot\frac{
                              D_{x+d}}{D_{x}}\right)\]
              където $N_x$ е кумулативната настояща стойност на бъдещите плащания, $N_{x+d}$ е кумулативната настояща стойност на бъдещите плащания след изтичане на гарантирания период от $n$ години, а $D_x$ е настоящата стойност на бъдещия паричен поток. Това представлява фактор, който се използва за изчисляване на месечната сума на Допълнителна пожизнена пенсия с разсрочено изплащане. Която се изчислява като:
              \[P = \frac{S-\sum_{i=1}^{m}T_i}{k_3}\]
              където $P$ е месечната сума на пенсията, $S$ е натрупаната сума в пенсионния фонд към момента на пенсиониране, а $T_m$ са месечните вноски по време на периода на разсрочено изплащане. Те се изчисляват като:
              \[T_i= H_i \cdot v^{(\frac{b_i}{12})}\]
              където $H_i$ е размера на месечното плащане повреме на разсроченият петиод, а $b_i$ е броят на месеците от началото на разсроченото изплащане до момента на съответното плащане.
\end{itemize}
\newpage
\section{Устройство}
Имплементирани са следните класове:
\begin{itemize}
        \item \textbf{Person} - представлява човек с атрибути като дата на раждане, пол и други.
        \item \textbf{PensionFund} - представлява пенсионен фонд с атрибути като име, тип (задължителен или доброволен), и други.
        \item \textbf{Company} - представлява компания, която предлага пенсионни продукти.
        \item \textbf{Calculator} - съдържа методи за изчисляване на пенсията въз основа на натрупаните вноски, лихвени проценти и други фактори.
        \item \textbf{Finances} - съдържа методи за финансови изчисления, като изчисляване на настояща стойност, бъдеща стойност и други.
        \item \textbf{main} - основен скрипт, който демонстрира използването на библиотеката.
\end{itemize}
И се използват следните примерни данни:
\begin{itemize}
        \item \textbf{NSI\_mortality\_table.csv} - таблица на смъртността от Националната осигурителна институция (НОИ).
        \item \textbf{Saglasie\_fund\_actives.csv} - данни за активите на пенсионен фонд "Съгласие".
\end{itemize}
Също така са използвани следните външни библиотеки:
\begin{itemize}
        \item \textbf{NumPy} - за числени изчисления и работа с масиви \cite{numpy2024}.
        \item \textbf{Pandas} - за обработка и анализ на данни \cite{pandas2024}.
        \item \textbf{Matplotlib} - за визуализация на данни \cite{matplotlib2024}.
        \item \textbf{Seaborn} - за статистическа визуализация на данни \cite{seaborn2024}.
        \item \textbf{Datetime} - за работа с дати и времена.
        \item \textbf{Math} - за математически функции и операции.
        \item \textbf{Random} - за генериране на случайни числа.
        \item \textbf{CSV} - за работа с CSV файлове.
\end{itemize}
Нека зарзгледаме основните класове и техните методи по-подробно:
\subsection{Company}
Клас, който създава обекти - компания със съответните параметри. Те включват:
\begin{itemize}
        \item Име на компанията
        \item Технически лихвен процент
        \item Риск при разсрочено изплащане
        \item База данни със стойностите на активите на компанията във различните фондове, като Универсален, Доброволен и Професионален Пенсионен фонд, подредени по дата.
        \item Хора - масив свъв който се добавят участниците във пенсионният фонд (обекти от класа Person)
        \item Общо хора ($l_0$) за актюерски изчисления, константа която по принцип се задава като равна на 100 000 за актюерските изчисления в българия.
\end{itemize}

\subsection{Person}
Клас, който създава обекти хора, които ще бъдат участници във фонда. Параметрите включват:
\begin{itemize}
        \item Име
        \item ЕГН
        \item Пол
        \item Възраст
        \item Доход
        \item Спестявания
        \item Осигурителен стаж
        \item Пенсионна възраст
        \item Предполагаеми спестявания при пенсиониране
\end{itemize}
Използва редица методи за преобразуването на ЕГН в дата на раждане и изчисляване на възрастта, както и намиране на пола на лицето. Както и оставащото време за работа до пенсиониране, което служи за намирането на предполагаемите спестявания при пенсиониране.
\subsection{Calculator}
Основния клас във работната среда. Съдържа методи за изчисляване на пенсията, базирани на входните данни от класовете Person и Company.
\newline
Съдържа класове за изчисляване на актюерските константи като: $q_x, p_x, l_x, v_x, D_x$ и $N_x$ по формулите описани в раздел Математика. Съдържа и клас от методи за извличане на информацията от базата данни от актюерски константи на НОИ. Освен това има и клас от функции - Pension с три подкласа изчисляващи факторите за трите пенсии и месечните плащания за всяка една от тях при всякакви условия.
\subsection{Finances}
Финансов клас, който управлява всички финансови операции и изчисления. Той включва методи за:
\begin{itemize}
        \item Извличане на информация от базата данни на компанията за стойността на активи във различните фондове.
        \item Пресмятане на плаващо средно за стойността на активите, което служи за анализ на тенденциите във времето и прогнозиране на бъдещи стойности.
        \item Изчисляване на средното и стандартното отклонение на движенията на активите. (Предполага се че следват аналогия на бял шум (White Noise))
        \item Симулация на бъдещи стойности на активите чрез Монте Карло метод.
        \item Изчисляване на бъдещата стойност на спестяванията при различни сценарии.
        \item Изобразяване на данните чрез графики.
\end{itemize}
\subsection{main}
Това е основният скрипт, който демонстрира използването на библиотеката. Той включва:
\begin{itemize}
        \item Създаване на обекти от класовете Company и Person с примерни данни.
        \item Извикване на методите от класовете Calculator и Finances за изчисляване на пенсията и анализ на финансовите данни.
        \item Визуализация на резултатите чрез графики.
\end{itemize}
\newpage

\section{Резултати}


\newpage
\section{Заключение}

\newpage
\begin{thebibliography}{99} % Use a high number (99) for many references

        % International/English Sources
        \bibitem[bostonfed1999]{bostonfed1999}
        Munnell, A. H. (1999).
        \newblock \emph{Demographic Changes and Funding for Pension Plans}.
        \newblock Boston Fed. Retrieved from: \url{https://www.bostonfed.org/-/media/Documents/conference/16/conf16b.pdf}
        \newblock Accessed: 2025-09-17.

        \bibitem[dalriada2023]{dalriada2023}
        Dalriada Trustees. (2023).
        \newblock \emph{The Importance of Mathematics in Pensions}.
        \newblock Retrieved from: \url{https://www.dalriadatrustees.co.uk/the-importance-of-mathematics-in-pensions/}
        \newblock Accessed: 2025-09-17.

        \bibitem[milliman2022]{milliman2022}
        Milliman. (2022).
        \newblock \emph{Using Python as an Actuarial Modelling Platform}.
        \newblock Retrieved from: \url{https://www.milliman.com/en/insight/using-python-actuarial-modelling-platform}
        \newblock Accessed: 2025-09-17.

        \bibitem[ncpers2023]{ncpers2023}
        National Conference on Public Employee Retirement Systems (NCPERS). (2023).
        \newblock \emph{The Impact of Demographic Shifts on Public Pensions (And What They Can Do About It)}.
        \newblock Retrieved from: \url{https://www.ncpers.org/blog_home.asp?display=377}
        \newblock Accessed: 2025-09-17.

        \bibitem[lifelib2024]{lifelib2024}
        Binette, O. (2024).
        \newblock \emph{lifelib: Actuarial Models in Python}.
        \newblock Retrieved from: \url{https://lifelib.io/}
        \newblock Accessed: 2025-09-17.

        \bibitem[hyperexponential2023]{hyperexponential2023}
        Hyperexponential. (2023).
        \newblock \emph{Python for Insurers and Actuaries}.
        \newblock Retrieved from: \url{https://www.hyperexponential.com/blog/python-for-insurance/}
        \newblock Accessed: 2025-09-17.

        % Bulgarian Sources
        \bibitem[ZDZPO2004]{ZDZPO_2004}
        Народно събрание на Република България. (2004).
        \newblock \emph{Закон за допълнителното задължително пенсионно осигуряване}.
        \newblock Доставен от: \url{https://www.parliament.bg/bg/laws}
        \newblock Последно посетен на: 18.09.2023 г.

        \bibitem[ZDOO2000]{ZDOO_2000}
        Народно събрание на Република България. (2000).
        \newblock \emph{Закон за държавното обществено осигуряване}.
        \newblock Доставен от: \url{https://www.parliament.bg/bg/laws}
        \newblock Последно посетен на: 18.09.2023 г.

        \bibitem[DKFNPensions]{DKFN_Pensions}
        Държавна комисия по финансов надзор (ДКФН). (2023).
        \newblock \emph{Раздел „Пенсионни фондове“}.
        \newblock Доставен от: \url{https://www.dkn.bg}
        \newblock Последно посетен на: 18.09.2023 г.

        \bibitem[NOIOfficial]{NOI_Official}
        Национална осигурителна институция (НОИ). (2023).
        \newblock \emph{Официален уебсайт}.
        \newblock Доставен от: \url{https://www.noi.bg}
        \newblock Последно посетен на: 18.09.2023 г.

        \bibitem[NumPy2024]{numpy2024}
        Harris, C. R., Millman, K. J., van der Walt, S. J. et al. (2024).
        \newblock \emph{Array programming with NumPy}.
        \newblock Nature, 585(7825), 357–362.
        \newblock Retrieved from: \url{https://www.numpy.org}
        \newblock Accessed: 2025-09-17.

        \bibitem[pandas2024]{pandas2024}
        The pandas development team. (2024).
        \newblock \emph{pandas-dev/pandas: Pandas}.
        \newblock Zenodo.
        \newblock Retrieved from: \url{https://pandas.pydata.org}
        \newblock Accessed: 2025-09-17.

        \bibitem[Matplotlib2024]{matplotlib2024}
        Hunter, J. D., and the Matplotlib development team. (2024).
        \newblock \emph{Matplotlib: Visualization with Python}.
        \newblock Retrieved from: \url{https://matplotlib.org}
        \newblock Accessed: 2025-09-17.

        \bibitem[Seaborn2024]{seaborn2024}
        Waskom, M. L., and the seaborn development team. (2024).
        \newblock \emph{Seaborn: statistical data visualization}.
        \newblock Journal of Open Source Software, 6(60), 3021.
        \newblock Retrieved from: \url{https://seaborn.pydata.org}
        \newblock Accessed: 2025-09-17.
\end{thebibliography}
\newpage
\subsection{Приложения}
\subsubsection{Код}
\lstinputlisting[language=Python, caption=main.py]{../Code/main.py}
\lstinputlisting[language=Python, caption=company.py]{../Code/Company.py}
\lstinputlisting[language=Python, caption=person.py]{../Code/Person.py}
\lstinputlisting[language=Python, caption=calculator.py]{../Code/Calculator.py}
\lstinputlisting[language=Python, caption=finances.py]{../Code/Finances.py}

\end{document}
