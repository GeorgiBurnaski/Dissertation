\documentclass[12pt]{article}
% Encoding and language support 
\usepackage[utf8]{inputenc} % Input encoding
\usepackage[T2A]{fontenc} % Font encoding for Cyrillic (Bulgarian)
\usepackage[bulgarian, english]{babel} % Language support

\usepackage{amsmath, amssymb}
\usepackage{listings} 
\usepackage{xcolor}
\usepackage{url}

%Comments
\usepackage{pdfcomment}

%Coding
\usepackage{listings}
\usepackage{xcolor} % Required for colors

% Define colors similar to Pygments/minted
\definecolor{codegreen}{rgb}{0,0.6,0}
\definecolor{codegray}{rgb}{0.5,0.5,0.5}
\definecolor{codepurple}{rgb}{0.58,0,0.82}



% Listing style
\lstset{
    commentstyle=\color{codegreen},
    keywordstyle=\color{magenta},
    numberstyle=\tiny\color{codegray},
    stringstyle=\color{codepurple},
    basicstyle=\ttfamily\footnotesize,
    breakatwhitespace=false,         
    breaklines=true,                 
    captionpos=b,                    
    keepspaces=true,                 
    numbers=left,                    
    numbersep=5pt,                  
    showspaces=false,                
    showstringspaces=false,
    showtabs=false,                  
    tabsize=2,
    frame=none,
}

\title{\textbf{Дисертация на тема
        \\[0.5em] \large{Разработване на виртуална среда за пенсионни актюерски изчисления в Python}}}
\author{Георги Веселинов Бурнаски
        \\ Катедра Приложна Математика
        \\ Факултет по Математика и Информатика
        \\ Софийски Университет "Св. Климент Охридски"}
\date{\today}

\begin{document}
\maketitle
\newpage
\section{Увод}
Актюерската математика заема все по-значимо място в съвременното общество, където управлението на финансовите рискове и дългосрочната устойчивост на социалните системи са от първостепенно значение.
\newline
\newline
В условията на застаряващо население, динамични финансови пазари и нарастващи изисквания към пенсионните системи, надеждните и прецизни актюерски изчисления се превръщат в ключов инструмент за вземане на информирани решения. Те подпомагат както държавните институции при моделиране и реформиране на пенсионните схеми, така и частните компании при разработване на застрахователни и инвестиционни продукти\cite{dalriada2023}.
\newline
\newline
Актюерските методи намират широко приложение в различни сфери: пенсионното осигуряване, животозастраховането здравното застраховане, управлението на фондове и финансовото планиране. Общото между всички тези области е необходимостта от точни прогнози, основани на статистически модели и вероятностни методи, които да оценяват бъдещи парични потоци, продължителност на живота и свързаните с тях рискове \cite{milliman2022}.
\newline
\newline
В наши дни пенсионните системи се изправят пред редица сериозни предизвикателства. Демографските промени, и по-специално увеличаването на средната продължителност на живота и намаляването на раждаемостта, водят до нарастващо съотношение между пенсионери и
активно работещо население \cite{bostonfed1999}. Това поставя под натиск публичните пенсионни фондове и създава необходимост от по-прецизни модели за оценка на бъдещите задължения. Паралелно с това колебанията на финансовите пазари и инфлационните процеси влияят върху доходността на пенсионните активи и изискват по-гъвкави подходи при управлението на риска \cite{ncpers2023}.
\newline
\newline
Съвременните технологии и програмни езици като Python предлагат нови възможности за прилагане на актюерската математика в практиката. Разработването на специализирана библиотека за пенсионни актюерски изчисления има двойна значимост: от една страна, улеснява изследователите и практиците при прилагането на сложни модели, а от друга – допринася за повишаване на прозрачността, възпроизводимостта и достъпността на тези изчисления \cite{lifelib2024, hyperexponential2023}.
\newline
\newline
Въпреки наличието на различни софтуерни решения за актюерски изчисления, много от тях са или твърде специализирани, или не предлагат необходимата гъвкавост и интеграция с други аналитични инструменти. Python, с богатия си екосистем от библиотеки за научни изчисления и машинно обучение, се явява като идеална платформа за разработване на такава библиотека. Тя би могла да обедини теоретичните основи на актюерската математика с практическите нужди на потребителите, предоставяйки лесен за използване и разширяем инструмент \cite{numpy2024, pandas2024}.
\newline
\newline
Разработването на библиотека за пенсионни актюерски изчисления в Python би могло да включва функции за изчисляване на настояща стойност на бъдещи парични потоци, моделиране на демографски и финансови рискове, симулации на сценарии и оптимизация на пенсионни стратегии. Освен това, интеграцията с други библиотеки за визуализация и анализ би позволила по-добро представяне и интерпретация на резултатите \cite{matplotlib2024, seaborn2024}.
\newline
\newline
В заключение, създаването на специализирана библиотека за пенсионни актюерски изчисления в Python представлява важна стъпка към модернизацията и усъвършенстването на актюерската практика. Тя би могла да подпомогне както академичните изследвания, така и практическите приложения, като предостави мощен и достъпен инструмент за анализ и управление на пенсионните системи в съвременния свят.
\newline
\newline
Пенсионната система в България следва модела на много развити страни и се състои
от три компонента, известни като "трите стълба":
\begin{itemize}
        \item Първи стълб: Държавно задължително пенсионно осигуряване (солидарност между поколенията). Това е държавната пенсия, която се финансира от текущите осигурителни вноски на работещите. Тя е задължителна за всички работещи.
        \item Втори стълб: Допълнително задължително пенсионно осигуряване (ДЗПО). Това е индивидуално натрупване на средства в избрани от самия осигурено лице пенсионни фондове. Той също е задължителен за хората, родени след 31.12.1959 г., които не са избрали да останат само в Първи стълб.
        \item Трети стълб: Допълнително доброволно пенсионно осигуряване (ДДПО). Това е напълно доброволно допълнително пенсионно осигуряване, при което хората сами решават да спестяват допълнително за своята пенсия.
\end{itemize}
В този труд ще се фокусираме върху изчисленията, свързани със втори и трети стълб от пенсионносигурителната система, тъй като те са поети от частните пенсионноосигурителни дружества. В България 20\% от доходите на работещите се отделят за пенсионно осигуряване, като 12.8\% отиват за първи стълб, а 5\% - за втори стълб. Важно е да се отбележи, че тези проценти могат да варират в зависимост от законодателството и икономическите условия. Останалите 2.2\% са за здравно осигуряване и трудови злополуки.\cite{zdzpo2024, zdoo2024}
\newline
\newline
Условията за пенсионните изчисления са регламентирани от различни нормативни актове, като основните от тях са:
\begin{itemize}
        \item Кодекс за социално осигуряване (КСО)
        \item Закон за допълнително задължително пенсионно осигуряване (ЗДЗПО)
        \item Закон за допълнително доброволно пенсионно осигуряване (ЗДОО)
        \item Наредба № Н-8 от 2004 г. за определяне на методиката за изчисляване на пенсиите
        \item Наредба № 3 от 2003 г. за условията и реда за изплащане на пенсиите от частните пенсионни фондове
\end{itemize}
Като изчисленията трябва да отговарят на изискванията, заложени наредба 69 от 2003 г. за определяне на методиката за изчисляване на пенсиите от частните пенсионни фондове.
\newline
\newline
Целта на настоящата дисертация е именно изгражданетона такава библиотека, която да обедини теоретичните основи на пенсионната математика с предимствата на съвременните програмни среди. Чрез нея се цели създаването на гъвкав инструмент, който да подкрепя както научната работа, така и практическите решения в областта на пенсионното осигуряване.
\newpage
\section{Устройство}


\newpage
\section{Резултати}


\newpage
\section{Дискусия}


\newpage
\section{Заключение}


\bibliographystyle{plain}
\bibliography{ref}

\end{document}
