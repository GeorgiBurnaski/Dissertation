\documentclass[12pt]{article}

% Encoding and language support 
\usepackage[utf8]{inputenc} % Input encoding
\usepackage[T2A]{fontenc} % Font encoding for Cyrillic (Bulgarian)
\usepackage[bulgarian, english]{babel} % Language support

\usepackage{amsmath, amssymb}
\usepackage{listings} 
\usepackage{xcolor}
\usepackage{url}

%Comments
\usepackage{pdfcomment}

%Coding
\usepackage{listings}
\usepackage{xcolor} % Required for colors

% Define colors similar to Pygments/minted
\definecolor{codegreen}{rgb}{0,0.6,0}
\definecolor{codegray}{rgb}{0.5,0.5,0.5}
\definecolor{codepurple}{rgb}{0.58,0,0.82}


% Listing style
\lstset{
    commentstyle=\color{codegreen},
    keywordstyle=\color{magenta},
    numberstyle=\tiny\color{codegray},
    stringstyle=\color{codepurple},
    basicstyle=\ttfamily\footnotesize,
    breakatwhitespace=false,         
    breaklines=true,                 
    captionpos=b,                    
    keepspaces=true,                 
    numbers=left,                    
    numbersep=5pt,                  
    showspaces=false,                
    showstringspaces=false,
    showtabs=false,                  
    tabsize=2,
    frame=none,
}

\title{\textbf{Дисертация на тема
        \\[0.5em] \large{Разработване на виртуална среда за пенсионни актюерски изчисления в Python}}}
\author{Георги Веселинов Бурнаски
        \\ Катедра Приложна Математика
        \\ Факултет по Математика и Информатика
        \\ Софийски Университет "Св. Климент Охридски"}
\date{\today}

\begin{document}
\maketitle
\newpage
\section{Увод}

Актюерската математика заема все по-значимо място в съвременното общество, където управлението на финансовите рискове и дългосрочната устойчивост на социалните системи са от първостепенно значение.
\newline
\newline
В условията на застаряващо население, динамични финансови пазари и нарастващи изисквания към пенсионните системи, надеждните и прецизни актюерски изчисления се превръщат в ключов инструмент за вземане на информирани решения. Те подпомагат както държавните институции при моделиране и реформиране на пенсионните схеми, така и частните компании при разработване на застрахователни и инвестиционни продукти\cite{dalriada2023}.
\newline
\newline
Актюерските методи намират широко приложение в различни сфери: пенсионното осигуряване, животозастраховането здравното застраховане, управлението на фондове и финансовото планиране. Общото между всички тези области е необходимостта от точни прогнози, основани на статистически модели и вероятностни методи, които да оценяват бъдещи парични потоци, продължителност на живота и свързаните с тях рискове \cite{milliman2022}.
\newline
\newline
В наши дни пенсионните системи се изправят пред редица сериозни предизвикателства. Демографските промени, и по-специално увеличаването на средната продължителност на живота и намаляването на раждаемостта, водят до нарастващо съотношение между пенсионери и
активно работещо население \cite{bostonfed1999}. Това поставя под натиск публичните пенсионни фондове и създава необходимост от по-прецизни модели за оценка на бъдещите задължения. Паралелно с това колебанията на финансовите пазари и инфлационните процеси влияят върху доходността на пенсионните активи и изискват по-гъвкави подходи при управлението на риска \cite{ncpers2023}.
\newline
\newline
Съвременните технологии и програмни езици като Python предлагат нови възможности за прилагане на актюерската математика в практиката. Разработването на специализирана библиотека за пенсионни актюерски изчисления има двойна значимост: от една страна, улеснява изследователите и практиците при прилагането на сложни модели, а от друга – допринася за повишаване на прозрачността, възпроизводимостта и достъпността на тези изчисления \cite{lifelib2024, hyperexponential2023}.
\newline
\newline
Целта на настоящата дисертация е именно изгражданетона такава библиотека, която да обедини теоретичните основи на пенсионната математика с предимствата на съвременните програмни среди. Чрез нея се цели създаването на гъвкав инструмент, който да подкрепя както научната работа, така
и практическите решения в областта на пенсионното осигуряване.
\newpage
\section{Устройство}


\newpage
\section{Резултати}


\newpage
\section{Дискусия}


\newpage
\section{Заключение}


\bibliographystyle{plain}
\bibliography{ref.bib}
\end{document}
